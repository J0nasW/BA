%
% ****
\chapter{Performance \& Auswertung}
\label{chap:erg}
% ****
%
	Ziel dieser Arbeit war es, auf Grundlage eines bereits bestehenden, neuronalen Netzes eine Anwendung des Reinforcement Learning zur Regelung dynamischer Systeme zu erschaffen. Das neuronale Netz ist dabei kein konventionell erzeugtes Netz einer künstlichen Intelligenz, sondern beruht auf biologischen Ereignissen.\\
	Somit wurde zuerst eine Berechnungsgrundlage eines solchen Netzes geschaffen und implementiert. Dazu wurden verschiedene numerische Lösungsverfahren von Differentialgleichungen verglichen und umgesetzt. Letztlich wird ein universeller Simulator geschaffen, welcher Informationen über die Nervenzellen und Synapsen erhält und entsprechend in der Lage ist, das gesamte Netz zu simulieren und Fire-Events auszugeben. Um die Performance des neuronalen Netzes durch den Simulator zu messen, wird eine Simulationsumgebung eingebunden und ein Lern-Algorithmus implementiert. In dieser Arbeit wird sich auf die Reinforcement Learning Methode RandomSearch konzentriert sowie auf die Optimierungsmethode durch Gewichten der entsprechenden Synapsen.

% ***
\section{Performance implementierter Algorithmen}
\label{sec:erg_performance}
% ***
	
	
	
% ***
\section{Limitationen und Alternativen von Algorithmen}
\label{sec:erg_lim}
% ***

% ***
\section{Vergleich zu bestehenden Systemen}
\label{sec:erg_vgl}
% ***

% ***
\section{Zusammenfassung}
\label{sec:erg_zsm}
% ***
	\begin{algorithm}
		\SetKwInOut{Input}{Input}
		\SetKwInOut{Output}{Output}
		
		\Input{$u, u_{rest}, t, \vartheta, R, C, I_0$}
		\Output{Array $u(t)$ mit $t=1,2,3,...$}
		
		\For{$i\leftarrow 0$ \KwTo $t_{max}$}{
		\emph{$i$ als Zähl-Variable}\\
			\eIf{$u \leq \vartheta$}{
				\tcc{Aufaddieren, bis der Threshold $\vartheta$ erreicht ist}
				Berechne momentane Spannung $u$ bei $t=i$\\
				Erweitere das Array $u_{array}$ um aktuelle Spannung $u$
				i hochzählen $i += 1$
			}{
				\tcc{Treshold $\vartheta$ ist erreicht, setze $u$ auf $0$ zurück}
				$i = 0$
				Berechne momentane Spannung $u$ bei $t=0$\\
				Erweitere das Array $u_{array}$ um aktuelle Spannung $u$
				i hochzählen $i += 1$
			}	
		}
		\KwRet{$u_{array}$}
		
		\caption{Das LIF-Modell}
	\end{algorithm}

%%% Local Variables: 
%%% mode: latex
%%% TeX-master: "main"
%%% End: 
