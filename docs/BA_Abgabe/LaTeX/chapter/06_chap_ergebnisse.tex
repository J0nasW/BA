%
% ****
\chapter{Performance \& Auswertung}
\label{chap:erg}
% ****
%

	Reinforcement Learning (kurz: RL) kann als einer der drei großen Bereiche des Maschine Learning interpretiert werden. Neben den Bereichen ``Supervised'' und ``Unsupervised Learning'' deckt es ein weites Spektrum an Anwendungsfeldern ab.\\
	\textit{Supervised Learning} (zu Deutsch: Überwachtes Lernen) bildet die Grundlage und wurde in den Anfängen der künstlichen Intelligenz eingesetzt. Ein Algorithmus lernt aus gegebenen Paaren von Ein- und Ausgängen eine Funktion, welche nach mehrmaligen Trainingsläufen Assoziationen herstellen soll und auf neue Eingaben passende Ausgaben liefert. 
		

% ***
\section{Performance implementierter Algorithmen}
\label{sec:erg_performance}
% ***
	
	
	
% ***
\section{Limitationen und Alternativen von Algorithmen}
\label{sec:erg_lim}
% ***

% ***
\section{Vergleich zu bestehenden Systemen}
\label{sec:erg_vgl}
% ***

% ***
\section{Zusammenfassung}
\label{sec:erg_zsm}
% ***
	\begin{algorithm}
		\SetKwInOut{Input}{Input}
		\SetKwInOut{Output}{Output}
		
		\Input{$u, u_{rest}, t, \vartheta, R, C, I_0$}
		\Output{Array $u(t)$ mit $t=1,2,3,...$}
		
		\For{$i\leftarrow 0$ \KwTo $t_{max}$}{
		\emph{$i$ als Zähl-Variable}\\
			\eIf{$u \leq \vartheta$}{
				\tcc{Aufaddieren, bis der Threshold $\vartheta$ erreicht ist}
				Berechne momentane Spannung $u$ bei $t=i$\\
				Erweitere das Array $u_{array}$ um aktuelle Spannung $u$
				i hochzählen $i += 1$
			}{
				\tcc{Treshold $\vartheta$ ist erreicht, setze $u$ auf $0$ zurück}
				$i = 0$
				Berechne momentane Spannung $u$ bei $t=0$\\
				Erweitere das Array $u_{array}$ um aktuelle Spannung $u$
				i hochzählen $i += 1$
			}	
		}
		\KwRet{$u_{array}$}
		
		\caption{Das LIF-Modell}
	\end{algorithm}

%%% Local Variables: 
%%% mode: latex
%%% TeX-master: "main"
%%% End: 
