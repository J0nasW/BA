%
% ****
\chapter*{Abstract und Kurzfassung}
\addcontentsline{toc}{chapter}{Abstract und Kurzfassung}
% ****
%

%
% ***
\section*{Abstract}
% ***
%

English version ... approx. $\frac{1}{2}$ page

%
% ***
\section*{Kurzfassung}
% ***
%

{\Large --- ARBEITSSTAND ---}\vspace{1cm}\\
Diese Bachelorarbeit umfasst den Ansatz, durch \textit{Reinforcement Learning} eine zuverlässige und vergleichbare Regelung von dynamischen Systemen zu erzielen, welche bisher nur durch klassische Regelansätze möglich gewesen ist. Dabei wird im ersten Schritt in Anlehnung an das Tierreich das neuronale Netz des Wurms C. Elegans \cite{WormLevelRL} genauer betrachtet. Der Touch-Withdrawal Circuit ist dafür zuständig, dass das Tier bei Berührung zurückschnellt. Diese Verschaltung der verschiedenen Nervenzellen mittels Synapsen und Gap-Junctions lassen sich exakt nachbilden und mathematisch durch das s.g. \textit{Leaky Integrate and Fire Model} beschreiben.\\
Weiterhin soll ein universeller Simulator zur Implementierung dieses Netzwerkes geschaffen werden, welcher durch zahlreiche Module in der Lage ist, geeignete Parameter zu finden und das Netzwerk auf Probleme der Regelungstechnik anzuwenden. Der Simulator wird dabei in der Programmiersprache \texttt{Python} realisiert und enthält mehrere Dependencies.\\
Ziel dieser Bachelorarbeit ist es, eine verlässliche Regelung dynamischer Systeme als Simulation zu erschaffen und diese mit konventionellen Ansätzen der Regelungstechnik qualitativ zu vergleichen.


%%% Local Variables: 
%%% mode: latex
%%% TeX-master: "main"
%%% End: 
