%
% ****
\chapter*{Abstract und Kurzfassung}
\addcontentsline{toc}{chapter}{Abstract und Kurzfassung}
% ****
%

%
% ***
\section*{Abstract}
% ***
%

English version ... approx. $\frac{1}{2}$ page

%
% ***
\section*{Kurzfassung}
% ***
%

Diese Bachelorarbeit umfasst den Ansatz, durch \textit{Reinforcement Learning} eine zuverlässige und vergleichbare Regelung von dynamischen Systemen zu erzielen, welche bisher nur durch klassische Regelansätze möglich gewesen ist. Dabei wird im ersten Schritt in Anlehnung an das Tierreich das neuronale Netz des Wurms C. Elegans \cite{WormLevelRL} genauer betrachtet, um Rückschlüsse auf die Lernalgorithmen zu erhalten. Die Verschaltung der verschiedenen Neuronen mittels Synapsen lassen sich so exakt mathematisch durch das s.g. \textit{Leaky Integrate and Fire Model} beschreiben und simulativ nachbauen.\\
Im zweiten Teil wird mittels der Programmiersprache \code{Python} ein neuronales Netz zur Regelung der Problemstellung des inversen Pendels implementiert und durch Reinforcement Learning Algorithmen trainiert. Hier kommen verschiedene Werkzeuge wie Python Libraries, TensorFlow und andere Hilfsmittel zum Einsatz.\\
Ziel dieser Bachelorarbeit ist es, eine verlässliche Regelung dynamischer Systeme als Simulation zu erschaffen und diese mit konventionellen Ansätzen der Regelungstechnik qualitativ zu vergleichen.









%%% Local Variables: 
%%% mode: latex
%%% TeX-master: "main"
%%% End: 
